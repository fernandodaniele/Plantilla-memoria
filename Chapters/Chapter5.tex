% Chapter Template

\chapter{Conclusiones} % Main chapter title

\label{Chapter5} % Change X to a consecutive number; for referencing this chapter elsewhere, use \ref{ChapterX}


%----------------------------------------------------------------------------------------

En este capítulo se destacan los objetivos cumplidos con el trabajo realizado y se plantean los pasos a seguir para realizar mejoras futuras.

%----------------------------------------------------------------------------------------
%	SECTION 1
%----------------------------------------------------------------------------------------

\section{Resultados obtenidos }

Los objetivos y requerimientos planteados al inicio del proyecto fueron cumplidos en su totalidad. El tiempo de ejecución de las tareas fue similar al estimado, pero no pudo respetarse el cronograma debido a las restricciones por el COVID 19. A pesar de la ocurrencia de estos riesgos, el único efecto fue el cambio de los plazos planificados y las tareas fueron ejecutadas a la normalidad en el tiempo estimado.

Además, es importante destacar el uso de los conocimientos y técnicas utilizadas en el desarrollo que fueron adquiridos en las asignaturas de la especialización:
\begin{itemize}
 \item Programación de microcontroladores: uso de máquinas de estados finitos en la interfaz de usuario.
 \item Sistemas operativos de tiempo real: uso de FreeRTOS para la ejecución de tareas con tiempo de ejecución crítico.
 \item Protocolos de comunicación en sistemas embebidos: uso de protocolos y comunicaciones, como ser UART, SPI y Wi-Fi.
 \item Diseño de circuitos impreso: uso de técnicas de diseño de esquemáticos y circuitos impresos.
\end{itemize}

Por último, es importante destacar que se obtuvo un prototipo funcional que cumple con las funciones básicas de los tituladores comerciales y es accesible económicamente para universidades y laboratorios pequeños.

%----------------------------------------------------------------------------------------
%	SECTION 2
%----------------------------------------------------------------------------------------
\section{Trabajo futuro}

El siguiente paso a seguir es realizar mejoras funcionales respecto al \textit{firmware}. Hay detalles a mejorar en la interfaz de usuario, por ejemplo la gráfica del pH respecto al tiempo debería ser de pH respecto al volumen. Además, se piensa agregar una opción para que el usuario pueda configurar la red Wi-Fi a la cual desea conectar el dispositivo. 

Las principales mejoras respecto al hardware dependen de la inclusión de nuevos componentes. La utilización de un sensor de temperatura permitiría ajustar el valor de pH para cuando se trabaje con muestras que no estén a temperatura ambiente. Además, el uso de un sensor de flujo permitiría tener una medida del volumen inyectado y crear un sistema de control de lazo cerrado.