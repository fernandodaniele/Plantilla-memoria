\chapter{Introducción específica} % Main chapter title

\label{Chapter2}

%----------------------------------------------------------------------------------------
%	SECTION 1
%----------------------------------------------------------------------------------------
En este capítulo se  realiza un revisión detallada de los dispositivos y tecnologías utilizados y que fueron desarrollados por terceros para comprender las decisiones de diseño tomadas que se mencionan en el capítulo \ref{Chapter3}. 

%----------------------------------------------------------------------------------------
\section{Electrodos de pH}
\label{sec:electrodoPH}

Un electrodo muy usado hoy en día, es el electrodo combinado que está formado por dos electrodos dentro del mismo encapsulado. En la figura \ref{fig:electrodoCombinado} se observa un electrodo combinado que está formado por un electrodo de vidrio que se encuentra dentro del tubo de vidrio interno, y un electrodo de referencia de Ag-AgCl que se encuentra dentro del tubo de epoxi externo, en una solución de KCl saturada.

\begin{figure}[htbp]
	\centering
	\includegraphics[width=.5\textwidth]{./Figures/electrodoCombinado.png}
	\caption{Electrodo combinado de pH de Ag/AgCl\protect\footnotemark.}
	\label{fig:electrodoCombinado}
\end{figure}

\footnotetext{Imagen tomada de \url{https://aulavirtual.agro.unlp.edu.ar/pluginfile.php/36460/mod_resource/content/4/pHmetro.\%20Clase\%20te\%C3\%B3rica\%202do.\%20Cuatrimestre\%202017..pdf}}

El potencial de un electrodo está dado por la ecuación de Nernst, que se puede escribir de manera simplificada como muestra la ecuación \ref{eq:Nernst} \citep{ARTICLE:4}.

\begin{equation}
	\label{eq:Nernst}
E = E^{O} + k pH
\end{equation}

donde E es el potencial corregido del electrodo, $E^{O}$ es el potencial en condiciones estándar (valores tabulados), k una variable que depende de la temperatura y pH es el valor de pH de la muestra.

En este trabajo se utilizó el electrodo comercial marca HANNA HI-1230B de la figura \ref{fig:hi-1230b}, definido por el área de Ingeniería Química, ya que permite realizar titulaciones potenciométricas ácido-base para detectar nitrógeno en suelo y alcalinidad en agua. Específicamente, este electrodo es de plata sumergido en una disolución de cloruro de potasio que se ha saturado con cloruro de plata, y presenta las características mostradas en la tabla \ref{tab:hi1230b}. 

\begin{figure}[htbp]
	\centering
	\includegraphics[width=.6\textwidth]{./Figures/hi-1230b.jpeg}
	\caption{Electrodo hi-1230b.jpeg.}
	\label{fig:hi-1230b}
\end{figure}

\begin{table}[h]
	\centering
	\caption[caption corto]{Especificaciones técnicas del electrodo HANNA HI-1230B a 25 °C}
	\begin{tabular}{l c}    
		\toprule
		\textbf{Característica} & \textbf{Valor} \\
		\midrule
		Offset 7,01 pH	& 0,0 mV \\
		Slope 4,01 pH	& 172,2 mV \\			
		Tolerancia		& $\pm$ 10 mV  \\
		\bottomrule
		\hline
	\end{tabular}
	\label{tab:hi1230b}
\end{table}

En base a estos datos se puede crear la recta que relaciona la potencial del electrodo con el valor de pH y que está dada por la ecuación \ref{eq:phElectrodo}:

\begin{equation}
	\label{eq:phElectrodo}
pH = -0,0174 E + 7,01
\end{equation}

donde E es el potencial entregado por el electrodo y pH es el valor correspondiente de pH de la muestra a 25 °C.

%----------------------------------------------------------------------------------------
\section{Bombas peristálticas}

\textit{ NOTA: ESTA SECCIÓN SE ENCUENTRA EN CONSTRUCCIÓN. VER SECCIÓN DE REQUERIMIENTOS}
En la figura \ref{fig:bombaPeristEsq} se muestra una bomba peristaltica.

\begin{figure}[htbp]
	\centering
	\includegraphics[width=.6\textwidth]{./Figures/bombaPeristEsq.png}
	\caption{Bomba peristáltica\protect\footnotemark.}
	\label{fig:bombaPeristEsq}
\end{figure}

\footnotetext{Imagen tomada de \url{https://www.researchgate.net/figure/Figura-3-Principio-de-funcionamiento-de-una-bomba-peristaltica-de-3-rodillos_fig2_275959587}}

%----------------------------------------------------------------------------------------
\section{Otras técnologías utilizadas}
\textit{ NOTA: ESTA SECCIÓN SE ENCUENTRA EN CONSTRUCCIÓN}

%----------------------------------------------------------------------------------------
\section{Requerimientos}
\label{sec:requerimientos}

En esta sección se detallan los requerimientos del sistema que fueron planteados en el plan trabajo, con ligeros cambios que surgieron durante el desarrollo del trabajo.

\begin{itemize}
\item Interfaces Externas
	\begin{itemize}
	\item El hardware debe contar con una pantalla TFT táctil. [TPA-ERH-01-REQ001]
	\item El hardware debe contar con un lector de tarjetas SD. [TPA-ERH-01-REQ002]
	\item El hardware debe contar con un driver para un motor paso a paso Nema 17. [TPA-ERH-01-REQ003]
	\item El hardware debe contar con una entrada para un electrodo de pH. [TPA-ERH-01-REQ004]
\end{itemize}
	
\item Funciones
	\begin{itemize}
	\item El usuario puede elegir mediante la pantalla táctil el volumen de corte de la titulación. [TPA-ERS-01-REQ001]
	\item El usuario puede elegir mediante la pantalla táctil si utilizar o no el agitador. Cuando el proceso de titulación comienza, el agitador debe activarse si así lo indicó el usuario. [TPA-ERS-01-REQ002]
	\item El usuario puede realizar mediante la pantalla táctil el proceso de calibración con cada uno de los tres buffers. [TPA-ERS-01-REQ003]
	\item Los valores de potencial obtenidos en el proceso de la calibración se deben guardar en la memoria flash del ESP32. [TPA-ERS-01-REQ004]
	\item El valor de pH se debe calcular de manera proporcional a la recta de ajuste de los valores de potencial obtenidos en la calibración. [TPA-ERS-01-REQ005]
	\item El usuario puede dar inicio al proceso de titulación mediante la pantalla táctil. [TPA-ERS-01-REQ006]
	\item Durante la titulación, la pantalla debe mostrar el valor actual leído en mV y en pH y una gráfica de pH en función del tiempo. [TPA-ERS-01-REQ007]
	\item Cada valor de volumen añadido junto al valor de potencial asociado durante el proceso de titulación deben almacenarse en un archivo de texto en la tarjeta sd. No es necesario que esto se haga en tiempo real. [TPA-ERS-01-REQ008]
	\item Cada valor de volumen añadido junto al valor de potencial asociado durante el proceso de titulación deben mostrarse en una página web almacenada en la memoria flash, una vez finalizada la titulación. [TPA-ERS-01-REQ009]
	\item El usuario puede acceder a la página web mediante una conexión Wi-Fi. No es necesario que esto se haga en tiempo real. [TPA-ERS-01-REQ010]
	\item El sistema debe ser capaz de leer y mostrar el potencial entregado por un electrodo de pH, con una resolución de 1 mV para la lectura del potencial y de 0,01 pH para su conversión a pH. [TPA-ERS-01-REQ011]
	\item El sistema deberá inyectar una cantidad de 0,1 mL esperar 5 segundos para realizar la medición de pH. La cantidad inyectada puede ser de 1 mL si el cambio de ph entre las últimas dos mediciones es menor a 0,2. [TPA-ERS-01-REQ012]
	 \item El sistema debe dejar de agregar titulante cuando se alcanza la cantidad de volumen indicada por el usuario como volumen de corte. [TPA-ERS-01-REQ013]

\end{itemize}

\item Requisitos de Rendimiento
	\begin{itemize}
	\item El sistema debe ser capaz de realizar titulaciones que involucren una cantidad máxima de 100 ml. [TPA-ERS-01-REQ014]
	\end{itemize}
	
\item Restricciones de Diseño
	\begin{itemize}
	\item Se utiliza el módulo ESP32 como computadora principal. [TPA-ERS-01-REQ015]
	\item Se utiliza la pantalla táctil MCU-FRIEND 2,4" como interfaz de usuario. [TPA-ERS-01-REQ016]
	\end{itemize}
\end{itemize}	
%----------------------------------------------------------------------------------------
% TODO LO DE ACÁ ABAJO ES EL MODELO

%\section{Estilo y convenciones}
%\label{sec:ejemplo}
%
%\subsection{Uso de mayúscula inicial para los título de secciones}
%
%Si en el texto se hace alusión a diferentes partes del trabajo referirse a ellas como capítulo, sección o subsección según corresponda. Por ejemplo: ``En el capítulo \ref{Chapter1} se explica tal cosa'', o ``En la sección \ref{sec:ejemplo} se presenta lo que sea'', o ``En la subsección \ref{subsec:ejemplo} se discute otra cosa''.
%
%Cuando se quiere poner una lista tabulada, se hace así:
%
%\begin{itemize}
%	\item Este es el primer elemento de la lista.
%	\item Este es el segundo elemento de la lista.
%\end{itemize}
%
%Notar el uso de las mayúsculas y el punto al final de cada elemento.
%
%Si se desea poner una lista numerada el formato es este:
%
%\begin{enumerate}
%	\item Este es el primer elemento de la lista.
%	\item Este es el segundo elemento de la lista.
%\end{enumerate}
%
%Notar el uso de las mayúsculas y el punto al final de cada elemento.
%
%\subsection{Este es el título de una subsección}
%\label{subsec:ejemplo}
%
%Se recomienda no utilizar \textbf{texto en negritas} en ningún párrafo, ni tampoco texto \underline{subrayado}. En cambio sí se debe utilizar \textit{texto en itálicas} para palabras en un idioma extranjero, al menos la primera vez que aparecen en el texto. En el caso de palabras que estamos inventando se deben utilizar ``comillas'', así como también para citas textuales. Por ejemplo, un \textit{digital filter} es una especie de ``selector'' que permite separar ciertos componentes armónicos en particular.
%
%La escritura debe ser impersonal. Por ejemplo, no utilizar ``el diseño del firmware lo hice de acuerdo con tal principio'', sino ``el firmware fue diseñado utilizando tal principio''. 
%
%El trabajo es algo que al momento de escribir la memoria se supone que ya está concluido, entonces todo lo que se refiera a hacer el trabajo se narra en tiempo pasado, porque es algo que ya ocurrió. Por ejemplo, "se diseñó el firmware empleando la técnica de test driven development".
%
%En cambio, la memoria es algo que está vivo cada vez que el lector la lee. Por eso transcurre siempre en tiempo presente, como por ejemplo:
%
%``En el presente capítulo se da una visión global sobre las distintas pruebas realizadas y los resultados obtenidos. Se explica el modo en que fueron llevados a cabo los test unitarios y las pruebas del sistema''.
%
%Se recomienda no utilizar una sección de glosario sino colocar la descripción de las abreviaturas como parte del mismo cuerpo del texto. Por ejemplo, RTOS (\textit{Real Time Operating System}, Sistema Operativo de Tiempo Real) o en caso de considerarlo apropiado mediante notas a pie de página.
%
%Si se desea indicar alguna página web utilizar el siguiente formato de referencias bibliográficas, dónde las referencias se detallan en la sección de bibliografía de la memoria, utilizado el formato establecido por IEEE en \citep{IEEE:citation}. Por ejemplo, ``el presente trabajo se basa en la plataforma EDU-CIAA-NXP \citep{CIAA}, la cual...''.
%
%\subsection{Figuras} 
%
%Al insertar figuras en la memoria se deben considerar determinadas pautas. Para empezar, usar siempre tipografía claramente legible. Luego, tener claro que \textbf{es incorrecto} escribir por ejemplo esto: ``El diseño elegido es un cuadrado, como se ve en la siguiente figura:''
%
%\begin{figure}[h]
%\centering
%\includegraphics[scale=.45]{./Figures/cuadradoAzul.png}
%\end{figure}
%
%La forma correcta de utilizar una figura es con referencias cruzadas, por ejemplo: ``Se eligió utilizar un cuadrado azul para el logo, como puede observarse en la figura \ref{fig:cuadradoAzul}''.
%
%\begin{figure}[ht]
%	\centering
%	\includegraphics[scale=.45]{./Figures/cuadradoAzul.png}
%	\caption{Ilustración del cuadrado azul que se eligió para el diseño del logo.}
%	\label{fig:cuadradoAzul}
%\end{figure}
%
%El texto de las figuras debe estar siempre en español, excepto que se decida reproducir una figura original tomada de alguna referencia. En ese caso la referencia de la cual se tomó la figura debe ser indicada en el epígrafe de la figura e incluida como una nota al pie, como se ilustra en la figura \ref{fig:palabraIngles}.
%
%\begin{figure}[htpb]
%	\centering
%	\includegraphics[scale=.3]{./Figures/word.jpeg}
%	\caption{Imagen tomada de la página oficial del procesador\protect\footnotemark.}
%	\label{fig:palabraIngles}
%\end{figure}
%
%\footnotetext{Imagen tomada de \url{https://goo.gl/images/i7C70w}}
%
%La figura y el epígrafe deben conformar una unidad cuyo significado principal pueda ser comprendido por el lector sin necesidad de leer el cuerpo central de la memoria. Para eso es necesario que el epígrafe sea todo lo detallado que corresponda y si en la figura se utilizan abreviaturas entonces aclarar su significado en el epígrafe o en la misma figura.
%
%
%
%\begin{figure}[ht]
%	\centering
%	\includegraphics[scale=.37]{./Figures/questionMark.png}
%	\caption{¿Por qué de pronto aparece esta figura?}
%	\label{fig:questionMark}
%\end{figure}
%
%Nunca colocar una figura en el documento antes de hacer la primera referencia a ella, como se ilustra con la figura \ref{fig:questionMark}, porque sino el lector no comprenderá por qué de pronto aparece la figura en el documento, lo que distraerá su atención.
%
%Otra posibilidad es utilizar el entorno \textit{subfigure} para incluir más de una figura, como se puede ver en la figura \ref{fig:three graphs}. Notar que se pueden referenciar también las figuras internas individualmente de esta manera: \ref{fig:1de3}, \ref{fig:2de3} y \ref{fig:3de3}.
% 
%\begin{figure}[!htpb]
%     \centering
%     \begin{subfigure}[b]{0.3\textwidth}
%         \centering
%         \includegraphics[width=.65\textwidth]{./Figures/questionMark}
%         \caption{Un caption.}
%         \label{fig:1de3}
%     \end{subfigure}
%     \hfill
%     \begin{subfigure}[b]{0.3\textwidth}
%         \centering
%         \includegraphics[width=.65\textwidth]{./Figures/questionMark}
%         \caption{Otro.}
%         \label{fig:2de3}
%     \end{subfigure}
%     \hfill
%     \begin{subfigure}[b]{0.3\textwidth}
%         \centering
%         \includegraphics[width=.65\textwidth]{./Figures/questionMark}
%         \caption{Y otro más.}
%         \label{fig:3de3}
%     \end{subfigure}
%        \caption{Tres gráficos simples}
%        \label{fig:three graphs}
%\end{figure}
%
%El código para generar las imágenes se encuentra disponible para su reutilización en el archivo \file{Chapter2.tex}.
%
%\subsection{Tablas}
%
%Para las tablas utilizar el mismo formato que para las figuras, sólo que el epígrafe se debe colocar arriba de la tabla, como se ilustra en la tabla \ref{tab:peces}. Observar que sólo algunas filas van con líneas visibles y notar el uso de las negritas para los encabezados.  La referencia se logra utilizando el comando \verb|\ref{<label>}| donde label debe estar definida dentro del entorno de la tabla.
%
%\begin{verbatim}
%\begin{table}[h]
%	\centering
%	\caption[caption corto]{caption largo más descriptivo}
%	\begin{tabular}{l c c}    
%		\toprule
%		\textbf{Especie}     & \textbf{Tamaño} & \textbf{Valor}\\
%		\midrule
%		Amphiprion Ocellaris & 10 cm           & \$ 6.000 \\		
%		Hepatus Blue Tang    & 15 cm           & \$ 7.000 \\
%		Zebrasoma Xanthurus  & 12 cm           & \$ 6.800 \\
%		\bottomrule
%		\hline
%	\end{tabular}
%	\label{tab:peces}
%\end{table}
%\end{verbatim}
%
%
%\begin{table}[h]
%	\centering
%	\caption[caption corto]{caption largo más descriptivo}
%	\begin{tabular}{l c c}    
%		\toprule
%		\textbf{Especie} 	 & \textbf{Tamaño} 		& \textbf{Valor}  \\
%		\midrule
%		Amphiprion Ocellaris & 10 cm 				& \$ 6.000 \\		
%		Hepatus Blue Tang	 & 15 cm				& \$ 7.000 \\
%		Zebrasoma Xanthurus	 & 12 cm				& \$ 6.800 \\
%		\bottomrule
%		\hline
%	\end{tabular}
%	\label{tab:peces}
%\end{table}
%
%En cada capítulo se debe reiniciar el número de conteo de las figuras y las tablas, por ejemplo, figura 2.1 o tabla 2.1, pero no se debe reiniciar el conteo en cada sección. Por suerte la plantilla se encarga de esto por nosotros.
%
%\subsection{Ecuaciones}
%\label{sec:Ecuaciones}
%
%Al insertar ecuaciones en la memoria dentro de un entorno \textit{equation}, éstas se numeran en forma automática  y se pueden referir al igual que como se hace con las figuras y tablas, por ejemplo ver la ecuación \ref{eq:metric}.
%
%\begin{equation}
%	\label{eq:metric}
%	ds^2 = c^2 dt^2 \left( \frac{d\sigma^2}{1-k\sigma^2} + \sigma^2\left[ d\theta^2 + \sin^2\theta d\phi^2 \right] \right)
%\end{equation}
%                                                        
%Es importante tener presente que si bien las ecuaciones pueden ser referidas por su número, también es correcto utilizar los dos puntos, como por ejemplo ``la expresión matemática que describe este comportamiento es la siguiente:''
%
%\begin{equation}
%	\label{eq:schrodinger}
%	\frac{\hbar^2}{2m}\nabla^2\Psi + V(\mathbf{r})\Psi = -i\hbar \frac{\partial\Psi}{\partial t}
%\end{equation}
%
%Para generar la ecuación \ref{eq:metric} se utilizó el siguiente código:
%
%\begin{verbatim}
%\begin{equation}
%	\label{eq:metric}
%	ds^2 = c^2 dt^2 \left( \frac{d\sigma^2}{1-k\sigma^2} + 
%	\sigma^2\left[ d\theta^2 + 
%	\sin^2\theta d\phi^2 \right] \right)
%\end{equation}
%\end{verbatim}
%
%Y para la ecuación \ref{eq:schrodinger}:
%
%\begin{verbatim}
%\begin{equation}
%	\label{eq:schrodinger}
%	\frac{\hbar^2}{2m}\nabla^2\Psi + V(\mathbf{r})\Psi = 
%	-i\hbar \frac{\partial\Psi}{\partial t}
%\end{equation}
%
%\end{verbatim}
